\part{Zarządzanie projektem}

\section{Założenia projektowe}
Celem projektu będzie zaprojektowanie i zbudowanie urządzenia służącego do nawigacji pieszej. Urządzenie będzie składało się z mikroprocesora z modułem Bluetooth, modułu GPS, wyświetlacza LCD oraz układu zasilającego. Konfiguracja będzie odbywała się za pomocą aplikacji mobilnej, w której możliwe będzie zaplanowanie trasy i wgranie jej do urządzenia.
Po skonfigurowaniu trasy, urządzenie będzie wskazywało kierunek i odległość do następnego punktu (na wyświetlaczu LCD).

\subsection{Wymagania projektowe}
Tablica MoSCoW

\section{Harmonogram}
tu jakiś tekst i opis

\subsection{Etap I - Projektowanie urządzenia i aplikacji}

\subsection{Etap II - Prototypowanie i testowanie}

\subsection{Etap III - Montaż}

\section{Ryzyka projektowe}

\section{Zarządzanie zespołem}
Sposób komunikacji, sposób planowania, podział zadań, itp. ?