\part{Zarządzanie projektem}

\section{Założenia projektowe}
Celem projektu będzie zaprojektowanie i zbudowanie urządzenia służącego do nawigacji pieszej. Urządzenie będzie składało się z mikroprocesora z modułem Bluetooth, modułu GPS, wyświetlacza LCD oraz układu zasilającego. Konfiguracja będzie odbywała się za pomocą aplikacji mobilnej, w której możliwe będzie zaplanowanie trasy i wgranie jej do urządzenia.
Po skonfigurowaniu trasy, urządzenie będzie wskazywało kierunek i odległość do następnego punktu (na wyświetlaczu LCD).

\subsection{Wymagania projektowe}
Wymagania projektowe zostały określone za pomocą tablicy MoSCoW:

\begin{itemize}
    \item \textbf{Must have:}
          \begin{itemize}
              \item Urządzenie poprawnie wskazuje kierunek i odległość do kolejnych punktów trasy.
              \item Aplikacja pozwala na zaplanowanie trasy na mapie i wgranie jej do pamięci urządzenia.
              \item Po konfiguracji urządzenie działa autonomicznie, bez potrzeby komunikacji z aplikacją.
          \end{itemize}
    \item \textbf{Should have:}
          \begin{itemize}
              \item Bateria pozwala na całodzienne korzystanie z urządzenia (około 8 godzin).
              \item Urządzenie implementuje mechanizmy zmniejszające zużycie energii (np. wygaszanie ekranu).
              \item Aktualny stan nawigacji jest zapisywany w pamięci nieulotnej, co umożliwia wznowienie korzystania z urządzenia po utracie zasilania.
          \end{itemize}
    \item \textbf{Could have:}
          \begin{itemize}
              \item Urządzenie zapisuje aktualne położenie i pozwala na zgranie przebiegu trasy do aplikacji.
              \item Aplikacja wyświetla historię przebytych tras razem z czasami przejścia.
              \item Realizacja urządzenia na samodzielnie zaprojektowanej płytce PCB.
          \end{itemize}
    \item \textbf{Won't have:}
          \begin{itemize}
              \item Urządzenie nie posiada interaktywnego interfejsu użytkownika, wyświetla jedynie aktualny stan trasy.
          \end{itemize}
\end{itemize}

\section{Ryzyka projektowe}
Realizacja projektu wiąże się z kilkoma ryzykami, które zostały zidentyfikowane i ocenione pod względem prawdopodobieństwa wystąpienia oraz wpływu na projekt. Poniżej przedstawiono główne ryzyka oraz planowane działania w celu ich minimalizacji.

\subsection{Odejście członków zespołu}
\begin{itemize}
    \item \textbf{Prawdopodobieństwo:} Bardzo niskie
    \item \textbf{Wpływ:} Bardzo niski
    \item \textbf{Plan działania:} Akceptacja ryzyka ze względu na niskie prawdopodobieństwo i wpływ.
\end{itemize}

\subsection{Ograniczony budżet}
\begin{itemize}
    \item \textbf{Prawdopodobieństwo:} Średnie
    \item \textbf{Wpływ:} Wysokie
    \item \textbf{Plan działania:} Opracowanie szczegółowego kosztorysu oraz pozyskanie dodatkowych środków finansowych od Politechniki.
\end{itemize}

\subsection{Opóźnienia w dostawie komponentów}
\begin{itemize}
    \item \textbf{Prawdopodobieństwo:} Średnie
    \item \textbf{Wpływ:} Średnie
    \item \textbf{Plan działania:} Uwzględnienie potencjalnych opóźnień w harmonogramie realizacji oraz opracowanie alternatywnych planów montażu.
\end{itemize}

\subsection{Opóźnienia w montażu układu (PCB)}
\begin{itemize}
    \item \textbf{Prawdopodobieństwo:} Średnie
    \item \textbf{Wpływ:} Średnie
    \item \textbf{Plan działania:} Uwzględnienie potencjalnych opóźnień w harmonogramie realizacji oraz opracowanie alternatywnego planu montażu.
\end{itemize}

\subsection{Niewystarczające umiejętności}
\begin{itemize}
    \item \textbf{Prawdopodobieństwo:} Średnie
    \item \textbf{Wpływ:} Średnie
    \item \textbf{Plan działania:} Akceptacja ryzyka oraz regularne szkolenia i konsultacje w zespole.
\end{itemize}

\subsection{Skomplikowana dokumentacja techniczna komponentów}
\begin{itemize}
    \item \textbf{Prawdopodobieństwo:} Średnie
    \item \textbf{Wpływ:} Niskie
    \item \textbf{Plan działania:} Wybór innych komponentów lub korzystanie z nieoficjalnych dokumentacji i wsparcia społeczności online.
\end{itemize}

\section{Harmonogram projektu}
Projekt został podzielony na trzy główne etapy:

\begin{itemize}
    \item \textbf{Etap I - Projektowanie} (do 8 kwietnia 2024)
          \begin{itemize}
              \item Opracowanie listy komponentów elektronicznych.
              \item Analiza i wybór komponentów.
              \item Projekt układu elektronicznego.
              \item Wybór technologii programowania aplikacji mobilnej.
              \item Projektowanie widoków aplikacji mobilnej.
              \item Opracowanie protokołu komunikacji między urządzeniem a aplikacją.
          \end{itemize}
    \item \textbf{Etap II - Prototypowanie i testowanie} (do 6 maja 2024)
          \begin{itemize}
              \item Zakup komponentów.
              \item Montaż układu na płytce prototypowej.
              \item Programowanie urządzenia i aplikacji mobilnej.
              \item Testy komunikacji między aplikacją a urządzeniem.
          \end{itemize}
    \item \textbf{Etap III - Montaż} (do 27 maja 2024)
          \begin{itemize}
              \item Montaż układu w wybranej technologii.
              \item Budowa obudowy urządzenia.
              \item Przygotowanie wersji produkcyjnej aplikacji mobilnej.
          \end{itemize}
\end{itemize}

\section{Zarządzanie zespołem}
\subsection{Podział zadań}
Przed rozpoczęciem projektu został ustalony następujący podział zadań:
\begin{itemize}
    \item \textbf{Dominik Cybulski, Michał Durkalec:} \\ Projektowanie układu elektronicznego, programowanie urządzenia.
    \item \textbf{Stanisław Kurzyp:} \\ Projektowanie aplikacji mobilnej.
\end{itemize}

\subsection{Planowanie pracy}
Do planowania poszczególnych zadań oraz monitorowania postępów wykorzystane zostało narzędzie Github Projects \cite{github-projects-docs}, które pozwala na
\begin{itemize}
    \item Tworzenie zadań i przypisywanie ich do członków zespołu.
    \item Planowanie terminów realizacji zadań.
    \item Monitorowanie postępów prac.
    \item Łączenie zadań z konkretnymi gałęziami kodu w systemie kontroli wersji.
\end{itemize}
