\part{Implementacja}

\section{Projekt urządzenia}
\subsection{Wybrane moduły elektroniczne}
Do budowy prototypu wybrano następujące moduły elektroniczne:
\begin{itemize}
    \item \textbf{MCU - Espressif DevKitV4 ESP-WROOM-32E}
          %   \begin{itemize}
          %       \item Dokumentacja: \url{https://www.espressif.com/sites/default/files/documentation/esp32-wroom-32e_esp32-wroom-32ue_datasheet_en.pdf}
          %   \end{itemize}
    \item \textbf{Magnetometr cyfrowy GY-273 3-osiowy I2C 3.3V / 5V - HMC5883L}
          %   \begin{itemize}
          %       \item Dokumentacja: \url{https://mm.digikey.com/Volume0/opasdata/d220001/medias/docus/635/HMC5883L.pdf}
          %   \end{itemize}
    \item \textbf{Moduł GPS GY-NEO6MV2 NEO-6M z Anteną}
          %   \begin{itemize}
          %       \item Dokumentacja: \url{https://www.u-blox.com/sites/default/files/products/documents/NEO-6_DataSheet_%28GPS.G6-HW-09005%29.pdf}
          %   \end{itemize}
    \item \textbf{Wyświetlacz LCD Waveshare 1.8inch 128x160}
          %   \begin{itemize}
          %       \item Dokumentacja: \url{https://files.waveshare.com/upload/e/e2/ST7735S_V1.1_20111121.pdf}
          %   \end{itemize}
\end{itemize}

\subsection{Kosztorys prototypu}
Kosztorys uwzględnia ceny głównych modułów urządzenia. Dotychczasowe koszty zostały pokryte przez członków grupy projektowej. Posiadamy również inne niezbędne części podstawowe, takie jak rezystory, kondensatory czy przyciski, które będą niezbędne przy testowaniu prototypu.

\begin{table}[h]
    \centering
    \begin{tabular}{|l|l|r|r|}
        \hline
        \textbf{Nazwa części} & \textbf{Typ}                 & \textbf{Cena detaliczna} & \textbf{Faktyczny koszt} \\ \hline
        Espressif DevKitV4    & ESP-WROOM-32E                & 35,99 zł                 & 35,99 zł                 \\ \hline
        GY-273                & Magnetometr cyfrowy 3-osiowy & 17,00 zł                 & 13,71 zł                 \\ \hline
        GY-NEO6MV2            & Moduł GPS z anteną           & 27,99 zł                 & 27,99 zł                 \\ \hline
        Waveshare 13892       & Wyświetlacz LCD 128x160px    & 33,90 zł                 & 0,00 zł                  \\ \hline
        \textbf{Razem}        &                              & \textbf{114,88 zł}       & \textbf{77,69 zł}        \\ \hline
    \end{tabular}
    \caption{Kosztorys prototypu}
\end{table}
\subsection{Napotkane problemy}
Tu można opisać problemy napotkane podczas projektowania urządzenia.

\section{Firmware}
Firmware aplikacji został napisany w języku C z wykorzystaniem platformy ESP-IDF \cite{esp-idf}.
ESP-IDF to oficjalne środowisko programistyczne dla platformy ESP32, które zawiera narzędzia do budowania, debugowania i wgrywania oprogramowania na urządzenie.

\subsection{Modularna struktura kodu} \label{sec:modularna-struktura-kodu}
Framework ESP-IDF pozwala na tworzenie projektów w sposób modułowy, co pozwala na łatwe dodawanie nowych funkcjonalności i niezależne testowanie poszczególnych komponentów.
W projekcie wyróżniono następujące moduły (komponenty):
\begin{itemize}
    \item \textbf{main} - moduł główny, inicjalizuje pozostałe moduły i definiuje zmienne współdzielone.
    \item \textbf{sc\_ble} - moduł odpowiedzialny za obsługę protokołu BLE. Zawiera definicje usług i charakterystyk GATT.
    \item \textbf{sc\_gps} - moduł obsługujący moduł GPS. Zawiera funkcje do odczytu współrzędnych geograficznych.
    \item \textbf{sc\_compass} - moduł obsługujący magnetometr. Zawiera funkcje do odczytu kierunku kompasu.
    \item \textbf{sc\_display} - moduł obsługujący wyświetlacz LCD. Zawiera funkcje do wyświetlania informacji na ekranie z wykorzystaniem biblioteki LVGL.
    \item \textbf{sc\_logic} - moduł odpowiedzialny za logikę aplikacji. Zawiera funkcje do obliczania kierunku i dystansu do celu.
    \item \textbf{lvgl} - moduł zawierający bibliotekę LVGL. \cite{lvgl-esp32-port} Zawiera konfigurację i funkcje pomocnicze do obsługi wyświetlacza.
    \item \textbf{lvgl\_esp32\_drivers} - moduł zawierający sterowniki dla ESP32 do biblioteki LVGL. \cite{lvgl-esp32-drivers}
\end{itemize}

\subsection{Architektura wielowątkowa}
Framework ESP-IDF jest oparty na architekturze FreeRTOS \cite{freertos-docs}, co pozwala na tworzenie wielowątkowych aplikacji.
Zawarta we frameworku wersja FreeRTOS pozwala na wykorzystanie dwóch rdzeni procesora ESP32, zapewniając równoległe wykonywanie zadań w dwóch wątkach.
Każdy z modułów aplikacji opisanych w sekcji \ref{sec:modularna-struktura-kodu} działa w osobnym wątku.

\subsubsection{Zmienne współdzielone}
Program zawiera dwie zmienne współdzielone odpowiedzialne za przechowywanie aktualnych danych na temat odczytów z sensorów oraz przechowywanie aktualnych danych do wyświetlenia na ekranie.

\begin{lstlisting}[language=C, caption=Definicja zmiennej dla odczytów sensorów]
typedef struct {
    SemaphoreHandle_t mutex;
    compass_position_t position;
    float bearing; // in radians
    float bearing_deg; // in degrees
    compass_path_t path;
    bool position_updated;
} compass_data_t;
\end{lstlisting}

\begin{lstlisting}[language=C, caption=Definicja zmiennej dla danych wynikowych]
typedef struct {
    SemaphoreHandle_t mutex;
    // Angle in 0.1 degrees - between 0 and 3600
    int16_t angle;
    // Next waypoint id
    uint16_t next_wp;
    // Distance to next waypoint in meters
    uint32_t distance;
    bool finished;
} display_data_t;
\end{lstlisting}

Każda zmienna posiada semafor, który zapewnia dostęp do zasobu w sposób atomowy, co zapobiega konfliktom między wątkami.



\subsection{Protokół komunikacji Bluetooth Low Energy (BLE)}
Protokół komunikacji BLE został zaimplementowany zgodnie ze standardem Bluetooth Core 4.2 \cite{bluetooth-core-spec-4-2} na platformie ESP32.
Użyto stosu i ESP-NimBLE \cite{esp-idf-nimble-docs}, który zapewni odpowiednie zarządzanie pamięcią oraz obsługę komunikacji BLE.
Stos ESP-NimBLE jest alternatywną wersją stosu Apache MyNewt Nimble \cite{apache-mynewt-docs}, przystosowaną do pracy z mikrokontrolerami producenta Espressif.

\subsubsection{Obsługa po stronie urządzenia}
Urządzenie wykorzystuje stos ESP-NimBLE do implementacji protokołu BLE, co pozwala na zmniejszenie zużycia pamięci.
Komunikacja z aplikacją mobilną opiera się na deklaracji usług i charakterystyk GATT, umożliwiając przesyłanie danych trasy w postaci tablic par współrzędnych geograficznych.

\begin{lstlisting}[language=C, caption=Definicja charakterystyki GATT]
static uint8_t gatt_svr_chr_val[MAX_CHAR_LEN];
static uint16_t gatt_svr_chr_val_handle;
static const ble_uuid128_t gatt_svr_chr_uuid =
        BLE_UUID128_INIT(0x00, 0x00, 0x00, 0x00, 0x11, 0x11, 0x11, 0x11,
                         0x22, 0x22, 0x22, 0x22, 0x33, 0x33, 0x33, 0x33);
\end{lstlisting}


\subsection{Napotkane problemy}
W trakcie rozwoju oprogramowania pojawiły się dwa główne problemy opisane w poniższych podrozdziałach.
\subsubsection{Konflikty między bibliotekami}
W trakcie rozwoju modułu odpowiedzialnego za komunikację z wyświetlaczem LCD pojawił się konflikt między:
\begin{itemize}
    \item Frameworkiem ESP-IDF
    \item Biblioteką LVGL,\cite{lvgl-esp32-port} która jest prostym silnikiem GUI
    \item Biblioteką LVGL ESP32 Drivers, \cite{lvgl-esp32-drivers} która zawiera sterowniki dla ESP32 do biblioteki LVGL
\end{itemize}
Początkowe próby integracji wszystkich trzech bibliotek w najnowszej wersji nie powiodły się,
co skutkowało koniecznością doboru kompatybilnych wersji bibliotek oraz dostosowaniem konfiguracji.
Proces naprawy konfliktów trwał kilka dni.

\subsubsection{Błąd w implementacji protokołu I2C w bibliotece ESP-IDF}
Do komunikacji z magnetometrem QMC5883 wykorzystano interfejs I2C \textit{(Inter Integrated Circut)}, który jest standardowym interfejsem komunikacyjnym dla urządzeń peryferyjnych.
Producent mikrokontrolera ESP32, firma Espressif, dostarcza bibliotekę do obsługi interfejsu I2C w ramach frameworku ESP-IDF.

W trakcie implementacji okazało się, że biblioteka ESP-IDF zawiera błąd w implementacji protokołu I2C, który uniemożliwiał poprawną komunikację z magnetometrem.
Dodatkowo, bład ten był niewykrywalny w procesie debugowania, co nie pozwalało na jednoznaczne zlokalizowanie problemu (mógł wynikać z błędu w kodzie, konfiguracji, podłęczeniu lub z uszkodzenia magnetometru).

Błąd naprawiono poprzez zastosowanie alternatywnej biblioteki do obsługi interfejsu I2C, co pozwoliło na poprawną komunikację z magnetometrem.

\section{Aplikacja mobilna}
\subsection{Napotkane problemy}
Tu można opisać problemy napotkane podczas pisania oprogramowania.